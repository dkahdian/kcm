% =============================
% Knowledge Compilation Map - Claims and Descriptions
% Auto-generated from database.json
% Generated: 2026-02-22T02:16:23.976Z
% 
% EDITING INSTRUCTIONS:
% - Claims (\begin{claim}...\end{claim}) are auto-generated. Do NOT edit.
% - Descriptions (\begin{claimdescription}...\end{claimdescription}) are EDITABLE.
% - Lines starting with "% [DERIVED" indicate auto-propagated edges.
% - To sync back to JSON, run: npx tsx scripts/latex-bijection.ts --to-json <this-file>
% =============================
\documentclass[11pt]{article}

% -------- Packages --------
\usepackage[margin=1in]{geometry}
\usepackage{amsmath, amssymb, amsthm}
\usepackage{mathtools}
\usepackage{enumitem}
\usepackage{hyperref}
\usepackage{cleveref}
\usepackage{xcolor}
\usepackage{natbib}

% -------- Hyperref setup --------
\hypersetup{
  colorlinks=true,
  linkcolor=blue,
  citecolor=blue,
  urlcolor=blue
}

% -------- Theorem styles --------
\theoremstyle{plain}
\newtheorem{theorem}{Theorem}[section]
\newtheorem{lemma}[theorem]{Lemma}
\newtheorem{proposition}[theorem]{Proposition}
\newtheorem{corollary}[theorem]{Corollary}
\newtheorem{claim}{Claim}[section]

\theoremstyle{definition}
\newtheorem{definition}[theorem]{Definition}
\newtheorem{example}[theorem]{Example}

\theoremstyle{remark}
\newtheorem{remark}[theorem]{Remark}

% -------- Description environment (just indented text, no prefix) --------
\newenvironment{claimdescription}{%
  \par\noindent\ignorespaces
}{\par}

% -------- Handy macros --------
\newcommand{\R}{\mathbb{R}}
\newcommand{\N}{\mathbb{N}}
\newcommand{\eps}{\varepsilon}

% -------- Title info --------
\title{Knowledge Compilation Map: Claims}
\date{\today}

\begin{document}
\maketitle

\tableofcontents
\newpage

% =============================
\section{Antoine Amarilli, Florent Capelli, Mikaël Monet, and Pierre Senellart, "Connecti...}
% Reference ID: Amarilli_2018
% =============================
\begin{claim}
$d$-$DNNF$ has unknown polynomial-time (but has quasi-polynomial-time) compilation to $uFBDD$ \citet{Amarilli_2018,Beame_2013}
\end{claim}
\begin{claimdescription}
The quasi-polynomial compilation from $dec$-$DNNF$ to $FBDD$ yields a $uFBDD$ when applied to a $d$-$DNNF$.
\end{claimdescription}

\begin{claim}
$d$-$SDNNF$ is polynomial-time compilable to $d$-$DNNF$ \citet{Amarilli_2018}
\end{claim}
\begin{claimdescription}
As defined by \citet{Amarilli_2018}, a $d$-$SDNNF$ is simply a structured $d$-$DNNF$, so compilation is trivial.
\end{claimdescription}

\begin{claim}
$d$-$SDNNF$ is polynomial-time compilable to $SDNNF$ \citet{Amarilli_2018}
\end{claim}
\begin{claimdescription}
As defined by \citet{Amarilli_2018}, $d$-$SDNNF$ is simply the subset of $SDNNF$ satisfying determinism.
\end{claimdescription}

\begin{claim}
$d$-$SDNNF$ has unknown polynomial-time (but has quasi-polynomial-time) compilation to $uOBDD$ \citet{Amarilli_2018,Bollig_2018,Beame_2015}
\end{claim}
\begin{claimdescription}
Theorem 2 of \citet{Bollig_2018} shows that $SDNNF$ can be compiled into $nOBDD$ by adapting the compilation of \citet{Beame_2015} from $DNNF$ to $nFBDD$; proposition 2 of \citet{Bollig_2018} shows that this $nOBDD$ is unambiguous if the $SDNNF$ is deterministic.
\end{claimdescription}

\begin{claim}
$dec$-$DNNF$ is polynomial-time compilable to $d$-$DNNF$ \citet{Amarilli_2018}
\end{claim}
\begin{claimdescription}
As defined by \citet{Amarilli_2018}, $dec$-$DNNF$ is the subset of $d$-$DNNF$ satisfying the decision property.
\end{claimdescription}

\begin{claim}
$dec$-$DNNF$ has unknown polynomial-time (but has quasi-polynomial-time) compilation to $FBDD$ \citet{Beame_2013}
\end{claim}
\begin{claimdescription}
Corollary 3.2 of \citet{Beame_2013} show quasi-polynomial compilation from $dec$-$DNNF$ to $FBDD$
\end{claimdescription}

\begin{claim}
$dec$-$SDNNF$ is polynomial-time compilable to $d$-$SDNNF$ \citet{Amarilli_2018}
\end{claim}
\begin{claimdescription}
As defined by \citet{Amarilli_2018}, $dec$-$SDNNF$ is the subset of $d$-$SDNNF$ satisfying the decision property, so compilation is trivial.
\end{claimdescription}

\begin{claim}
$dec$-$SDNNF$ is polynomial-time compilable to $dec$-$DNNF$ \citet{Amarilli_2018}
\end{claim}
\begin{claimdescription}
As defined by \citet{Amarilli_2018}, a $dec$-$SDNNF$ is simply a structured $dec$-$DNNF$, so discarding the v-tree completes the compilation.
\end{claimdescription}

\begin{claim}
$dec$-$SDNNF$ is not polynomial-time compilable to $nFBDD$ \citet{Amarilli_2018}
\end{claim}
\begin{claimdescription}
Shown by Proposition 3.10 of \citet{Amarilli_2018}
\end{claimdescription}

\begin{claim}
$dec$-$SDNNF$ has unknown polynomial-time (but has quasi-polynomial-time) compilation to $OBDD$ \citet{Amarilli_2018,Bollig_2018,Beame_2015}
\end{claim}
\begin{claimdescription}
Theorem 2 of \citet{Bollig_2018} shows that $SDNNF$ can be compiled into $nOBDD$ by adapting the compilation of \citet{Beame_2015} from $DNNF$ to $nFBDD$; proposition 2 of \citet{Bollig_2018} shows that this $nOBDD$ is unambiguous if the $SDNNF$ is deterministic. Additionally, the same compilation yields an $OBDD$ if the input is a $dec$-$SDNNF$ since there are no $\vee$-gates that are not decision gates in a $dec$-$SDNNF$.
\end{claimdescription}

\begin{claim}
$DNF$ is polynomial-time compilable to $nOBDD$ \citet{Amarilli_2018,Wegener_2000}
\end{claim}
\begin{claimdescription}
Fix any variable order $<$. Each term of a $DNF$ can be represented as an $OBDD_<$ of linear size (a chain of decision nodes for each literal in the term). The nondeterministic disjunction of these $OBDD_<$s yields an $nOBDD$ of size $O(|\phi|)$, since the variable order is shared across all branches \citet{Amarilli_2018,Wegener_2000}.
\end{claimdescription}

\begin{claim}
$FBDD$ is polynomial-time compilable to $dec$-$DNNF$ \citet{Amarilli_2018}
\end{claim}
\begin{claimdescription}
Rewrite each decision node labeled $x$ as $(x \wedge D_0) \vee (\neg x \wedge D_1)$, where $D_0$ and $D_1$ are the rewritings of the 0- and 1-successors. The introduced $\vee$-gates are decision gates and the $\wedge$-gates are decomposable, so all $\vee$-gates arise from decision node rewrites, yielding a $dec$-$DNNF$ in linear time \citet{Amarilli_2018}.
\end{claimdescription}

\begin{claim}
$FBDD$ is polynomial-time compilable to $uFBDD$ \citet{Amarilli_2018}
\end{claim}
\begin{claimdescription}
As defined by \citet{Amarilli_2018}, $FBDD$ is the subset of $uFBDD$ with no $\vee$-nodes.
\end{claimdescription}

\begin{claim}
$nFBDD$ is not quasi-polynomial-time compilable to $d$-$DNNF$ \citet{Amarilli_2018}
\end{claim}
\begin{claimdescription}
The Sauerhoff function has an $nFBDD$ of size $O(n^2)$ over $n^2$ variables, but as a $d$-$DNNF$ has size $2^{\Omega(n)}$.
\end{claimdescription}

\begin{claim}
$nFBDD$ is polynomial-time compilable to $DNNF$ \citet{Amarilli_2018}
\end{claim}
\begin{claimdescription}
Rewrite each decision node labeled $x$ as $(x \wedge D_0) \vee (\neg x \wedge D_1)$, where $D_0, D_1$ are the rewritings of the 0- and 1-successors. The $\wedge$-gates are decomposable, yielding a $DNNF$ in linear time \citet{Amarilli_2018}.
\end{claimdescription}

\begin{claim}
$nFBDD$ is not quasi-polynomial-time compilable to $FBDD$ \citet{Amarilli_2018}
\end{claim}
\begin{claimdescription}
Consider the Boolean functions $\psi_n,\phi_n:\{0,1\}^{n^2}\to\{0,1\}$ that respectively test whether, in an $n \times n$ Boolean matrix, either the number of 1's is odd and there is a full row of 1's, or the number of 1's is even and there is a column full of 1's. An $FBDD$ of size $O(n^2)$ may simulate either $\psi_n$ or $\phi_n$, and thus the $uFBDD$ $\psi_n\vee\phi_n$ is of size $O(n^2)$; however, an $FBDD$ for $\psi_n\vee\phi_n$ is provably of size $2^{\Omega(n^{1/2})}$ \citet{Amarilli_2018}.
\end{claimdescription}

\begin{claim}
$nFBDD$ is not quasi-polynomial-time compilable to $uFBDD$ \citet{Amarilli_2018}
\end{claim}
\begin{claimdescription}
The Sauerhoff function has an $nFBDD$ of size $O(n^2)$ over $n^2$ variables, but as a $d$-$DNNF$, which generalize $uFBDD$, has size $2^{\Omega(n)}$.
\end{claimdescription}

\begin{claim}
$nOBDD$ is polynomial-time compilable to $nFBDD$ \citet{Amarilli_2018}
\end{claim}
\begin{claimdescription}
As defined by \citet{Amarilli_2018}, $nOBDD$ is the subset of $nFBDD$ satisfying an order of the variables, so compilation entails discarding the order.
\end{claimdescription}

\begin{claim}
$nOBDD$ is polynomial-time compilable to $SDNNF$ \citet{Amarilli_2018}
\end{claim}
\begin{claimdescription}
Construct a right-linear v-tree from the variable order, then rewrite each decision node labeled $x$ as $(x \wedge D_0) \vee (\neg x \wedge D_1)$, where $D_0, D_1$ are the rewritings of the 0- and 1-successors. The v-tree alignment ensures every $\wedge$-gate respects the variable partition, yielding an $SDNNF$ in linear time \citet{Amarilli_2018}.
\end{claimdescription}

\begin{claim}
$nOBDD$ is not quasi-polynomial-time compilable to $uOBDD$ \citet{Amarilli_2018}
\end{claim}
\begin{claimdescription}
Any $DNF$ can be represented as an $nOBDD$ in linear time; however the family of $DNF$s $(\phi_n)_{n\in\N}$ built from expander graphs have treewidth linear in their size. Thus, any $uOBDD$ for $\phi_n$ is of size $2^{\Omega(|\phi_n|)}$. (TODO: write intuitive explanation or define expander graph separator)
\end{claimdescription}

\begin{claim}
$OBDD$ is polynomial-time compilable to $dec$-$SDNNF$ \citet{Amarilli_2018}
\end{claim}
\begin{claimdescription}
Construct a right-linear v-tree from the variable order, then rewrite each decision node labeled $x$ as $(x \wedge D_0) \vee (\neg x \wedge D_1)$, where $D_0, D_1$ are the rewritings of the 0- and 1-successors. The v-tree alignment ensures structuredness, and since all $\vee$-gates arise from decision node rewrites, every $\vee$-gate is a decision gate, yielding a $dec$-$SDNNF$ in linear time \citet{Amarilli_2018}.
\end{claimdescription}

\begin{claim}
$OBDD$ is polynomial-time compilable to $uOBDD$ \citet{Amarilli_2018}
\end{claim}
\begin{claimdescription}
As defined by \citet{Amarilli_2018}, $OBDD$ is the subset of $uOBDD$ without any $\vee$-node.
\end{claimdescription}

\begin{claim}
$SDNNF$ is polynomial-time compilable to $DNNF$ \citet{Amarilli_2018}
\end{claim}
\begin{claimdescription}
As defined by \citet{Amarilli_2018}, a structured $DNNF$ is a triple of the corresponding $DNNF$, the v-tree $T$, and a mapping $\rho$; as such, extracting a $DNNF$ from a $SDNNF$ is trivial.
\end{claimdescription}

\begin{claim}
$SDNNF$ has unknown polynomial-time (but has quasi-polynomial-time) compilation to $nOBDD$ \citet{Bollig_2018,Beame_2015}
\end{claim}
\begin{claimdescription}
Theorem 2 of \citet{Bollig_2018} shows that $SDNNF$ can be compiled into $nOBDD$ by adapting the compilation of \citet{Beame_2015} from $DNNF$ to $nFBDD$.
\end{claimdescription}

\begin{claim}
$uFBDD$ is polynomial-time compilable to $d$-$DNNF$ \citet{Amarilli_2018}
\end{claim}
\begin{claimdescription}
Rewrite each decision node labeled $x$ as $(x \wedge D_0) \vee (\neg x \wedge D_1)$, where $D_0, D_1$ are the rewritings of the 0- and 1-successors. The $\wedge$-gates are decomposable, and unambiguity of the $uFBDD$ ensures all introduced $\vee$-gates are deterministic, yielding a $d$-$DNNF$ in linear time \citet{Amarilli_2018}.
\end{claimdescription}

\begin{claim}
$uFBDD$ is not quasi-polynomial-time compilable to $FBDD$ \citet{Bova_2016,Amarilli_2018,Wegener_2000}
\end{claim}
\begin{claimdescription}
Consider the Boolean functions $\phi_n, \psi_n: \{0,1\}^{n^2}\to\{0,1\}$, which respectively test whether, in an $n\times n$ Boolean matrix, that the number of 1's is odd and there is a row full of 1's, and that the number of 1's is even and there is a column full of 1's. \citet{Bova_2016} and \citet{Wegener_2000} show that an $FBDD$ expression for $\phi_n \vee \psi_n$ has necessarily size $2^{\Omega(n^{1/2})}$; but since an $FBDD$ of size $O(n^2)$ may test $\phi_n$ or $\psi_n$ separately (shown by Proposition 3.1 of \citet{Amarilli_2018}), a $uFBDD$ for $\phi_n\vee \psi_n$ may be obtained by simply adding an $\vee$-gate joining these two $FBDD$ expressions, since $\phi_n \wedge \psi_n$ is unsatisfiable.
\end{claimdescription}

\begin{claim}
$uFBDD$ is polynomial-time compilable to $nFBDD$ \citet{Amarilli_2018}
\end{claim}
\begin{claimdescription}
As defined by \citet{Amarilli_2018}, $uFBDD$ is the subset of $nFBDD$ satisfying unambiguity.
\end{claimdescription}

\begin{claim}
$uOBDD$ is polynomial-time compilable to $d$-$SDNNF$ \citet{Amarilli_2018}
\end{claim}
\begin{claimdescription}
Construct a right-linear v-tree from the variable order, then rewrite each decision node labeled $x$ as $(x \wedge D_0) \vee (\neg x \wedge D_1)$, where $D_0, D_1$ are the rewritings of the 0- and 1-successors. The v-tree alignment ensures structuredness, and unambiguity of the $uOBDD$ makes all introduced $\vee$-gates deterministic, yielding a $d$-$SDNNF$ in linear time \citet{Amarilli_2018}.
\end{claimdescription}

\begin{claim}
$uOBDD$ is polynomial-time compilable to $nOBDD$ \citet{Amarilli_2018}
\end{claim}
\begin{claimdescription}
As defined by \citet{Amarilli_2018}, $uOBDD$ is the subset of $nOBDD$ satisfying unambiguity.
\end{claimdescription}

\begin{claim}
$uOBDD$ is not quasi-polynomial-time compilable to $OBDD$ \citet{Amarilli_2018}
\end{claim}
\begin{claimdescription}
The hidden weighted bit function $HWB_n$ has polynomial $uOBDD$ size, but $OBDD$ size $2^{\Omega(n)}$ \citet{Amarilli_2018}.
\end{claimdescription}

\begin{claim}
$uOBDD$ is polynomial-time compilable to $uFBDD$ \citet{Amarilli_2018}
\end{claim}
\begin{claimdescription}
As defined by \citet{Amarilli_2018}, $uOBDD$ is the subset of $uFBDD$ satisfying an order of the variables, so compilation entails discarding the order.
\end{claimdescription}


% =============================
\section{A. Darwiche and P. Marquis, "A Knowledge Compilation Map," Journal of Artificial...}
% Reference ID: Darwiche_2002
% =============================
\begin{claim}
$CNF$ is polynomial-time compilable to $NNF$ \citet{Darwiche_2002}
\end{claim}
\begin{claimdescription}
As defined by \citet{Darwiche_2002}, $CNF$ is a subset of $NNF$. Therefore, a polynomial compilation is trivial.
\end{claimdescription}

\begin{claim}
$CNF$ is not quasi-polynomial-time compilable to $PI$ \citet{Darwiche_2002}
\end{claim}
\begin{claimdescription}
It is well-known that some $CNF$ formulas (i.e. the negation of the parity function $\neg PARITY_n=\neg\bigoplus_{i=1}^n x_i$) have exponentially many prime implicants.
\end{claimdescription}

\begin{claim}
$d$-$DNNF$ is polynomial-time compilable to $DNNF$ \citet{Darwiche_2002}
\end{claim}
\begin{claimdescription}
As formalized by \citet{Darwiche_2002} and originally introduced in \citet{Darwiche_2001}, $d$-$DNNF$ is a subset of $DNNF$ satisfying the determinism property, making a polynomial compilation trivial.
\end{claimdescription}

\begin{claim}
$d$-$DNNF$ is not quasi-polynomial-time compilable to $MODS$ \citet{Darwiche_2002}
\end{claim}
\begin{claimdescription}
The $OR_n$ function $\bigvee_{i=1}^nx_i$ can be represented in $O(n)$ as a $d$-$DNNF$ formula but has $2^n-1$ models.
\end{claimdescription}

\begin{claim}
$DNF$ is not polynomial-time compilable to $d$-$DNNF$ unless the polynomial hierarchy collapses \citet{Darwiche_2002}
\end{claim}
\begin{claimdescription}
Because counting the models of a $d$-$DNNF$ is in P and counting the models of a $DNF$ is \#P-complete, $DNF$ cannot be polynomially compiled into $d$-$DNNF$ unconditionally. Such a compilation implies P = \#P, collapsing the polynomial hierarchy \citep{Darwiche_2002}.
\end{claimdescription}

\begin{claim}
$DNF$ is polynomial-time compilable to $DNNF$ \citet{Darwiche_2002}
\end{claim}
\begin{claimdescription}
As defined by \citet{Darwiche_2002}, $DNF$ is a subset of $DNNF$. Therefore, a polynomial compilation is trivial.
\end{claimdescription}

\begin{claim}
$DNF$ is not quasi-polynomial-time compilable to $IP$ \citet{Darwiche_2002}
\end{claim}
\begin{claimdescription}
It is well-known that some $DNF$ formulas (i.e. the parity function $PARITY_n=\bigoplus_{i=1}^n x_i$) have exponentially many prime implicates.
\end{claimdescription}

\begin{claim}
$DNNF$ has unknown polynomial-time (but has quasi-polynomial-time) compilation to $nFBDD$ \citet{Bodlaender_1993}
\end{claim}
\begin{claimdescription}
Section 5 of \citet{Bodlaender_1993} extends the compilation of $dec$-$DNNF$s into $FBDD$s (shown in \citet{Beame_2013}) to show that $DNNF$ can be quasi-polynomial-time compiled into $nFBDD$.
\end{claimdescription}

\begin{claim}
$DNNF$ is polynomial-time compilable to $NNF$ \citet{Darwiche_2002}
\end{claim}
\begin{claimdescription}
As defined by \citet{Darwiche_2002}, $DNNF$ is a subset of $NNF$, so a polynomial compilation is trivial.
\end{claimdescription}

\begin{claim}
$FBDD$ is polynomial-time compilable to $d$-$DNNF$ \citet{Darwiche_2002}
\end{claim}
\begin{claimdescription}
As defined by \citet{Darwiche_2002}, $FBDD$ is a subset of $d$-$DNNF$, so a polynomial compilation is trivial.
\end{claimdescription}

\begin{claim}
$IP$ is not quasi-polynomial-time compilable to $CNF$ \citet{Darwiche_2002}
\end{claim}
\begin{claimdescription}
Consider the $DNF$ formula $\Pi_n=\bigvee_{i=0}^{n-1}(x_{2i} \wedge x_{2i+1})$. This formula is in prime implicants form and each clause in $\Pi_n$ is an essential prime implicant of it. Hence its negation $\neg\Pi_n\in IP$.

Since the early work of Quine, we know that the number of essential prime implicates of a formula is a lower bound on the number of clauses that can be found in any $CNF$ representation of it. However, since $\Pi_n$ has $2^n$ prime implicants, $\neg\Pi_n$ has $2^n$ prime implicates, so any $CNF$ representation of $\neg\Pi_n$ must have at least $2^n$ clauses.
\end{claimdescription}

\begin{claim}
$IP$ is polynomial-time compilable to $DNF$ \citet{Darwiche_2002}
\end{claim}
\begin{claimdescription}
As defined by \citet{Darwiche_2002}, $IP$ is a subset of $DNF$. Therefore, a polynomial compilation is trivial.
\end{claimdescription}

\begin{claim}
$IP$ is not quasi-polynomial-time compilable to $FBDD$ \citet{Darwiche_2002,Wegener_1987}
\end{claim}
\begin{claimdescription}
The negation of the clique detection function $\neg\Sigma_{n,k}$ has $O(n^3)$ prime implicants (hence polynomial $IP$ size), but any $FBDD$ for $\neg\Sigma_{n,k}$ has exponential size for suitable $k$ \citet{Darwiche_2002,Wegener_1987}.
\end{claimdescription}

\begin{claim}
$IP$ is not quasi-polynomial-time compilable to $MODS$ \citet{Darwiche_2002}
\end{claim}
\begin{claimdescription}
The $OR_n$ function $\bigvee_{i=1}^nx_i$ can be represented in $O(n)$ as a $IP$ formula but has $2^n-1$ models.
\end{claimdescription}

\begin{claim}
$MODS$ is polynomial-time compilable to $CNF$ \citet{Darwiche_2002}
\end{claim}
\begin{claimdescription}
A Shannon tree for any $MODS$ formula can be generated in polynomial time. Paths from the root to 0-leaves yield a $CNF$ representation \citet{Darwiche_2002}.
\end{claimdescription}

\begin{claim}
$MODS$ is polynomial-time compilable to $d$-$DNNF$ \citet{Darwiche_2002}
\end{claim}
\begin{claimdescription}
As defined by \citet{Darwiche_2002}, $MODS$ is a subset of $d$-$DNNF$. Therefore, a polynomial compilation is trivial.
\end{claimdescription}

\begin{claim}
$MODS$ is polynomial-time compilable to $DNF$ \citet{Darwiche_2002}
\end{claim}
\begin{claimdescription}
As defined by \citet{Darwiche_2002}, $MODS$ is a subset of $DNF$ (any $MODS$ formula can be written as a disjunction of the individual models). Therefore, a polynomial compilation is trivial.
\end{claimdescription}

\begin{claim}
$MODS$ is polynomial-time compilable to $OBDD_<$ \citet{Darwiche_2002}
\end{claim}
\begin{claimdescription}
A Shannon tree for any $MODS$ formula following order $<$ can be generated in polynomial time. This tree can be reduced into a corresponding $OBDD_<$ in polynomial time \citet{Darwiche_2002}.
\end{claimdescription}

\begin{claim}
$OBDD$ is polynomial-time compilable to $FBDD$ \citet{Darwiche_2002}
\end{claim}
\begin{claimdescription}
As defined by \citet{Darwiche_2002}, $OBDD$ is a subset of $FBDD$. Therefore a polynomial time compilation is trivial.
\end{claimdescription}

\begin{claim}
$OBDD$ is not quasi-polynomial-time compilable to $OBDD_<$ \citet{Darwiche_2002}
\end{claim}
\begin{claimdescription}
The function $\bigwedge_{i=1}^n(x_i\Leftrightarrow y_i)$ has an $OBDD_<$ representation of size polynomial in $n$ whenever $<$ satisfies $x_1<y_1<...<x_n<y_n$ but exponential in $n$ if $<$ satisfies $x_1<x_2<...<x_n<y_1<y_2<...<y_n$
\end{claimdescription}

\begin{claim}
$OBDD_<$ is not quasi-polynomial-time compilable to $CNF$ \citet{Darwiche_2002}
\end{claim}
\begin{claimdescription}
The parity function $PARITY_n=\bigoplus_{i=1}^n x_i$ has linear size $OBDD_<$ representations but only exponential size $CNF$ representations.
\end{claimdescription}

\begin{claim}
$OBDD_<$ is not quasi-polynomial-time compilable to $DNF$ \citet{Darwiche_2002}
\end{claim}
\begin{claimdescription}
The parity function $PARITY_n=\bigoplus_{i=1}^n x_i$ has linear size $OBDD_<$ representations but only exponential size $DNF$ representations.
\end{claimdescription}

\begin{claim}
$OBDD_<$ is polynomial-time compilable to $OBDD$ \citet{Darwiche_2002}
\end{claim}
\begin{claimdescription}
As defined by \citet{Darwiche_2002}, $OBDD_<$ is a subset of $OBDD$. Therefore, a polynomial compilation is trivial.
\end{claimdescription}

\begin{claim}
$PI$ is polynomial-time compilable to $CNF$ \citet{Darwiche_2002}
\end{claim}
\begin{claimdescription}
As defined by \citet{Darwiche_2002}, $PI$ is a subset of $CNF$. Therefore, a polynomial compilation is trivial.
\end{claimdescription}

\begin{claim}
$PI$ is not quasi-polynomial-time compilable to $DNF$ \citet{Darwiche_2002}
\end{claim}
\begin{claimdescription}
The pairwise CNF $\Sigma_n = \bigwedge_{i=0}^{n-1}(x_{2i} \vee x_{2i+1})$ has $2^n$ prime implicates; its negation $\neg\Sigma_n \in PI$ therefore has $2^n$ essential prime implicants, forcing any $DNF$ representation to have at least $2^n$ terms \citet{Darwiche_2002}.
\end{claimdescription}

\begin{claim}
$PI$ is not quasi-polynomial-time compilable to $FBDD$ \citet{Wegener_1987,Darwiche_2002}
\end{claim}
\begin{claimdescription}
The clique detection function $\Sigma_{n,k}$ has $O(n^3)$ prime implicates (hence polynomial $PI$ size), but any $FBDD$ for $\Sigma_{n,k}$ has exponential size for suitable $k$ \citet{Wegener_1987,Darwiche_2002}.
\end{claimdescription}

\begin{claim}
$PI$ is not quasi-polynomial-time compilable to $MODS$ \citet{Darwiche_2002}
\end{claim}
\begin{claimdescription}
The $OR_n$ function $\bigvee_{i=1}^nx_i$ can be represented in $O(n)$ as a $PI$ formula but has $2^n-1$ models.
\end{claimdescription}


% =============================
\section{Paul Beame, Jerry Li, Sudeepa Roy, and Dan Suciu, "Model Counting of Query Expre...}
% Reference ID: Beame_2013
% =============================
\begin{claim}
$d$-$DNNF$ is not quasi-polynomial-time compilable to $dec$-$DNNF$ \citet{Beame_2013}
\end{claim}
\begin{claimdescription}
Shown by Corollary 3.5 of \citet{Beame_2013}.
\end{claimdescription}


% =============================
\section{Beate Bollig and Matthias Buttkus, "On the Relative Succinctness of Sentential D...}
% Reference ID: Bollig_2018
% =============================
\begin{claim}
$SDD$ is not quasi-polynomial-time compilable to $FBDD$ \citet{Bollig_2018}
\end{claim}
\begin{claimdescription}
The weighted sum function $WS_n$ can be represented by SDDs of size $O(n^3)$ (Corollary 3 of \citet{Bollig_2018}), but has exponential $FBDD$ size. Therefore, $P(SDD) \not\subseteq P(FBDD)$, which combined with the known result $P(FBDD) \not\subseteq P(SDD)$ from \citet{Beame_2015}, establishes that $SDD$ and $FBDD$ are incomparable (Corollary 4 of \citet{Bollig_2018}).
\end{claimdescription}


% =============================
\section{S. Bova, F. Capelli, S. Mengel, and F. Slivovsky, "Knowledge Compilation Meets C...}
% Reference ID: Bova_2016
% =============================
\begin{claim}
$DNNF$ is not quasi-polynomial-time compilable to $d$-$DNNF$ \citet{Bova_2016}
\end{claim}
\begin{claimdescription}
Proposition 7 of \citet{Bova_2016} shows that the Sauerhoff function $S_n$ has $DNNF$ size $O(n^2)$. Theorem 9 shows that $S_n$ has $d$-$DNNF$ size $2^{\Omega(n)}$.
\end{claimdescription}

\begin{claim}
$PI$ is not quasi-polynomial-time compilable to $DNNF$ \citet{Bova_2016}
\end{claim}
\begin{claimdescription}
Proposition 11 of \citet{Bova_2016} shows that the $JS_n$ has $PI$ size $O(n^2)$. Proposition 12 of \citet{Bova_2016} shows that $JS_n$ has $DNNF$ size $2^{\Omega(n^2)}$.
\end{claimdescription}


% =============================
\section{Ingo Wegener, "The Complexity of Boolean Functions," 1987.}
% Reference ID: Wegener_1987
% =============================
\begin{claim}
$MODS$ is polynomial-time compilable to $IP$ \citet{Wegener_1987}
\end{claim}
\begin{claimdescription}
The Quine-McCluskey algorithm generates an $IP$ formula in polynomial time.
\end{claimdescription}


% =============================
\section{G. Van den Broeck and A. Darwiche, "On the Role of Canonicity in Knowledge Compi...}
% Reference ID: Van den Broeck_2015
% =============================
\begin{claim}
$cSDD$ is polynomial-time compilable to $SDD$ \citet{Van den Broeck_2015}
\end{claim}
\begin{claimdescription}
As defined by \citet{Van den Broeck_2015}, $cSDD$ is the compressed subset of $SDD$, so compilation is trivial.
\end{claimdescription}

\begin{claim}
$SDD$ is polynomial-time compilable to $d$-$SDNNF$ \citet{Van den Broeck_2015}
\end{claim}
\begin{claimdescription}
$SDD$ is a strict subset of $d$-$SDNNF$.
\end{claimdescription}


% =============================
\section{S. Bova, "SDDs Are Exponentially More Succinct than OBDDs," Proceedings of the A...}
% Reference ID: Bova_2016_a
% =============================
% separators=Generalized_HWB
\begin{claim}
$cSDD$ is not quasi-polynomial-time compilable to $OBDD$ \citet{Bova_2016_a}
\end{claim}
\begin{claimdescription}
Theorem 4 of \citet{Bova_2016_a} constructs a generalized hidden weighted bit function $F_n$ on $2n+1$ variables with $cSDD$ size $O(n^3)$ but $OBDD$ size $2^{\Omega(n)}$. The function $F_n$ augments $HWB_n$ with fresh variables $y_0, \ldots, y_n$ to ensure the resulting SDD is compressed; conditioning $F_n$ on $y_i = 1$ recovers $HWB_n$, preserving Bryant's exponential $OBDD$ lower bound.
\end{claimdescription}


% =============================
\section{Adnan Darwiche, "On the tractable counting of theory models and its application ...}
% Reference ID: Darwiche_2000
% =============================
\begin{claim}
$d$-$DNNF$ is not quasi-polynomial-time compilable to $FBDD$ \citet{Darwiche_2000}
\end{claim}
\begin{claimdescription}
$d$-$DNNF$s can polynomially represent the disjunction of two functions with conflicting variable orders, which forces $FBDD$s to grow exponentially to track the state of both orders within a single "read-once" path.
\end{claimdescription}


% =============================
\section{Harry Vinall-Smeeth, "Structured d-DNNF Is Not Closed Under Negation," 2024.}
% Reference ID: Vinall-Smeeth_2024
% =============================
\begin{claim}
$d$-$SDNNF$ is not polynomial-time compilable to $SDD$ \citet{Vinall-Smeeth_2024}
\end{claim}
\begin{claimdescription}
Theorem 1 of \citet{Vinall-Smeeth_2024} proved that there exists a family of functions $f_n$ with $d$-$SDNNF$ of size $n$ but any $SDD$ representing $f_n$ has size $n^{\Omega(log(n))}$.
\end{claimdescription}


% =============================
\section{Florent Capelli, "Structural Restrictions of CNF-formulas: Applications to Model...}
% Reference ID: Capelli_2016
% =============================
\begin{claim}
$FBDD$ is not quasi-polynomial-time compilable to $SDNNF$ \citet{Capelli_2016,Pipatsrisawat_2010}
\end{claim}
\begin{claimdescription}
Appendix D.2 of \citet{Pipatsrisawat_2010} and Section 6.3 of \citet{Capelli_2016} independently proved this result using different techniques.
\end{claimdescription}


% =============================
\section{J. Gergov and C. Meinel, "Efficient Boolean manipulation with OBDD's can be exte...}
% Reference ID: Gergov_1994
% =============================
\begin{claim}
$FBDD$ is not quasi-polynomial-time compilable to $OBDD$ \citet{Gergov_1994}
\end{claim}
\begin{claimdescription}
The hidden weighted bit function $HWB_n$ has $OBDD$ size $2^{\Omega(n/4)}$ \citet{Bryant_1986} but $FBDD$ size $O(n^2)$ \citet{Gergov_1994}.
\end{claimdescription}


% =============================
\section{N. S. Kaleyski, "Boolean methods in knowledge compilation," 2016.}
% Reference ID: Kaleyski_2016
% =============================
\begin{claim}
$MODS$ is not polynomial-time compilable to $PI$ \citet{Kaleyski_2016}
\end{claim}
\begin{claimdescription}
\citet{Kaleyski_2016} constructs a family of Boolean functions $\psi_i$ with $MODS$ size $\Theta(2^{2i})$ but $PI$ size $\Omega(2^{(i-1)^2/4})$, ruling out a polynomial compilation. Whether an exponential separation exists remains open.
\end{claimdescription}


% =============================
\section{A. Darwiche, "SDD: a new canonical representation of propositional knowledge bas...}
% Reference ID: Darwiche_2011
% =============================
\begin{claim}
$OBDD$ is polynomial-time compilable to $cSDD$ \citet{Darwiche_2011}
\end{claim}
\begin{claimdescription}
As established by \citet{Darwiche_2011}, every $OBDD$ is a right-linear $SDD$, which is already compressed (canonical). Thus, $cSDD$ strictly generalizes $OBDD$.
\end{claimdescription}

% =============================
% Bibliography
% =============================
\bibliographystyle{plainnat}
\bibliography{refs}

\end{document}
