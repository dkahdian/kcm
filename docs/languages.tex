% =============================
% Knowledge Compilation Map - Language Definitions
% Auto-generated from database.json
% Generated: 2026-02-22T02:16:23.979Z
% 
% EDITING INSTRUCTIONS:
% - Language names in brackets are auto-generated. Do NOT edit.
% - Full names (\textbf{...}) are auto-generated. Do NOT edit.
% - Definition content (after the full name line) is EDITABLE.
% - To sync back to JSON, run: npx tsx scripts/latex-bijection.ts --to-json
% =============================
\documentclass[11pt]{article}

% -------- Packages --------
\usepackage[margin=1in]{geometry}
\usepackage{amsmath, amssymb, amsthm}
\usepackage{mathtools}
\usepackage{enumitem}
\usepackage{hyperref}
\usepackage{cleveref}
\usepackage{xcolor}
\usepackage{natbib}

% -------- Hyperref setup --------
\hypersetup{
  colorlinks=true,
  linkcolor=blue,
  citecolor=blue,
  urlcolor=blue
}

% -------- Theorem styles --------
\theoremstyle{definition}
\newtheorem{definition}{Definition}

% -------- Handy macros --------
\newcommand{\R}{\mathbb{R}}
\newcommand{\N}{\mathbb{N}}
\newcommand{\eps}{\varepsilon}

% -------- Title info --------
\title{Knowledge Compilation Map: Language Definitions}
\date{\today}

\begin{document}
\maketitle

\begin{definition}[$ANF$]\label{def:lang_8f666aa0}
\textbf{Algebraic Normal Form} \\
A Boolean formula represented exclusively using the exclusive-OR (XOR) and AND logical operators, taking the shape of a XOR sum of conjunctions. Originally introduced as Zhegalkin polynomials, it is highly structured but generally less space-efficient for standard compilation tasks compared to decision graphs \citet{Zhegalkin_1927}.
\end{definition}

\begin{definition}[$BDD$]\label{def:lang_bb65ddb5}
\textbf{Binary Decision Diagram} \\
A directed acyclic graph where each internal node represents a true/false decision on a specific boolean variable. Paths from the root to the leaf nodes explicitly map sequential variable assignments to a final 0 or 1 output \citet{Lee_1959}.
\end{definition}

\begin{definition}[$CNF$]\label{def:lang_89649e36}
\textbf{Conjunctive Normal Form} \\
A standard logical representation where a formula is an AND of clauses, with each clause being an OR of literals. While it is the foundational input format for modern SAT solvers, it supports very few polynomial-time operations natively \citet{Boole_1847}.
\end{definition}

\begin{definition}[$cSDD$]\label{def:lang_83e3b023}
\textbf{Canonical/Compressed Sentential Decision Diagram} \\
A Sentential Decision Diagram is called \emph{compressed}(\emph{canonical}) if, at each decision node, no two primes have the same sub \citet{Darwiche_2011}.
\end{definition}

\begin{definition}[$cSDD_T$]\label{def:lang_82fa749e}
\textbf{Canonical/Compressed Sentential Decision Diagram (wrt a fixed variable tree)} \\
The set of all cSDD formulas that respect a predefined variable tree (vtree) $T$. \citet{Darwiche_2011}.
\end{definition}

\begin{definition}[$d$-$DNNF$]\label{def:lang_6c130090}
\textbf{Deterministic Decomposable Negation Normal Form} \\
A variant of DNNF where the children of every OR node represent mutually exclusive (deterministic) formulas. This added determinism is the critical property that enables efficient, linear-time model counting and probabilistic inference \citet{Darwiche_2001a}.
\end{definition}

\begin{definition}[$d$-$SDNNF$]\label{def:lang_ea9b5299}
\textbf{-} \\
A deterministic DNNF where the variable decomposition across the graph is governed by a hierarchical variable tree (vtree). \citet{Pipatsrisawat_2008}.
\end{definition}

\begin{definition}[$d-SDNNF_T$]\label{def:lang_91f812d0}
\textbf{-} \\
The set of all d-SDNNF formulas that follow a fixed vtree $T$ \citet{Pipatsrisawat_2008}.
\end{definition}

\begin{definition}[$dec$-$DNNF$]\label{def:lang_981b62f0}
\textbf{Decision Decomposable Negation Normal Form} \\
A subset of DNNF where standard logical OR nodes are entirely replaced by decision nodes, similar to those found in BDDs. Because it relies explicitly on variable conditioning, it serves as the natural compilation trace of exhaustive DPLL-based SAT solvers \citet{Oztok_2014}.
\end{definition}

\begin{definition}[$dec$-$SDNNF$]\label{def:lang_0f27d539}
\textbf{Structured Decision Decomposable Negation Normal Form} \\
A decision DNNF whose underlying variable decisions strictly follow a hierarchical variable tree (vtree). This tree-driven structure standardizes the compilation trace and bridges the gap to sentential decision diagrams \citet{Oztok_2014}.
\end{definition}

\begin{definition}[$dec-SDNNF_<$]\label{def:lang_4ae03bc8}
\textbf{Structured Decision Decomposable Negation Normal Form (wrt a fixed variable tree)} \\
The set of all dec-SDNNF$_T$ formulas that follow a fixed vtree $T$. \citet{Oztok_2014}.
\end{definition}

\begin{definition}[$DNF$]\label{def:lang_4c204bf3}
\textbf{Disjunctive Normal Form} \\
A logical formula expressed as an OR of terms, where each term is an AND of literals.
\end{definition}

\begin{definition}[$DNNF$]\label{def:lang_3bebcab7}
\textbf{Decomposable Negation Normal Form} \\
The subset of NNF where the children of every AND node operate on completely disjoint sets of variables. This non-overlapping variable property, known as decomposability, guarantees polynomial-time consistency checking \citet{Darwiche_2001a}.
\end{definition}

\begin{definition}[$FBDD$]\label{def:lang_684b1ca7}
\textbf{Free Binary Decision Diagram} \\
A Binary Decision Diagram that allows variables to be tested in any order along any given path, provided no variable is tested more than once per path. This property is called the read-once property \citet{Gergov_1994}.
\end{definition}

\begin{definition}[$IP$]\label{def:lang_6ae90adc}
\textbf{Prime Implicants} \\
A formula represented exactly as the disjunction of all its minimal entailing conjunctions (prime implicants). \citet{Quine_1952}.
\end{definition}

\begin{definition}[$MODS$]\label{def:lang_e02902d0}
\textbf{Models} \\
The set of satisfying truth assignments for a given formula. It may be interpreted as the disjunction of each model, and thus as a d-SDNNF formula \citet{Darwiche_2002}.
\end{definition}

\begin{definition}[$nFBDD$]\label{def:lang_1df07cc3}
\textbf{Nondeterministic Free Binary Decision Diagram} \\
A generalization of the FBDD that allows for nondeterministic "guess" nodes in addition to standard variable decisions. \citet{Wegener_2000}.
\end{definition}

\begin{definition}[$NNF$]\label{def:lang_5bf00851}
\textbf{Negation Normal Form} \\
A boolean formula constructed strictly with AND, OR, and literals, where negations are only allowed directly on the variables. It serves as the broad, foundational structural superset for most tractable knowledge compilation languages \citet{Darwiche_2002}.
\end{definition}

\begin{definition}[$nOBDD$]\label{def:lang_d24efe0e}
\textbf{nondeterministic Ordered Binary Decision Diagram} \\
A generalization of the OBDD that permits nondeterministic branching while still enforcing a global linear variable ordering for its standard decision nodes. \citet{Wegener_2000}.
\end{definition}

\begin{definition}[$OBDD$]\label{def:lang_b9d72a7c}
\textbf{Ordered Binary Decision Diagram} \\
A Binary Decision Diagram where all variable decisions must follow a strict, uniform linear order $<$ across every path in the graph. They are strongly canonical and support powerful polytime Boolean operations, making them ubiquitous in formal hardware verification \citet{Bryant_1986}. \citet{Darwiche_2002}
\end{definition}

\begin{definition}[$OBDD_<$]\label{def:lang_d69995dd}
\textbf{Ordered Binary Decision Diagram (wrt a fixed variable order)} \\
The set of all OBDD formulas following a fixed variable order $<$, which is known in advance. This strictness allows the equivalence of two distinct functions to be checked in constant time \citet{Bryant_1986}. \citet{Bryant_1986}
\end{definition}

\begin{definition}[$PI$]\label{def:lang_27fffab2}
\textbf{Prime Implicates} \\
A formula represented exactly as the conjunction of all its minimal entailed disjunctive clauses (prime implicates). \citet{Quine_1952}.
\end{definition}

\begin{definition}[$SDD$]\label{def:lang_1afefbe2}
\textbf{Sentential Decision Diagram} \\
A representation that generalizes OBDDs by replacing binary decision nodes with "primes" and "subs", which are subcircuits also represented as SDDs. At each decision node, there is a set of (mutually exclusive and exhaustive) "primes", each with a corresponding "sub". Evaluating a decision node entails checking which prime is satisfied by the input variables, then evaluating that prime's sub. The SDD respects a vtree $T$ in the same way as an SDNNF \citet{Darwiche_2011}.
\end{definition}

\begin{definition}[$SDD_T$]\label{def:lang_9c84a267}
\textbf{Sentential Decision Diagram (wrt a fixed variable tree)} \\
The set of all SDD formulas respecting a fixed variable tree (vtree) $T$. \citet{Darwiche_2011}.
\end{definition}

\begin{definition}[$SDNNF$]\label{def:lang_b13b0d78}
\textbf{Structured Decomposable Negation Normal Form} \\
A DNNF that respects a vtree $T$, in the sense that the decomposability of variables across AND nodes is strictly governed by the hierarchical structure of $T$. This guarantees that variable partitions occur systematically \citet{Pipatsrisawat_2008}.
\end{definition}

\begin{definition}[$SDNNF_T$]\label{def:lang_3c803ba1}
\textbf{Structured Decomposable Negation Normal Form (wrt a fixed variable tree)} \\
The set of all SDNNF formulas respecting a fixed vtree $T$. \citet{Pipatsrisawat_2008}.
\end{definition}

\begin{definition}[$uFBDD$]\label{def:lang_4e62a038}
\textbf{Unambiguous Free Binary Decision Diagram} \\
A restricted form of a nondeterministic FBDD where every satisfying assignment has exactly one valid accepting path. This "unambiguity" preserves the ability to perform efficient model counting while allowing for greater compression \citet{Wegener_2000}.
\end{definition}

\begin{definition}[$uOBDD$]\label{def:lang_c2df8c2b}
\textbf{Unambiguous Ordered Binary Decision Diagram} \\
A nondeterministic OBDD restricted to unambiguous choices, maintaining efficient counting capabilities while strictly following a global variable order. It provides a middle ground between the strictness of OBDDs and the compression of nondeterminism \citet{Wegener_2000}.
\end{definition}

% =============================
% Bibliography
% =============================
\bibliographystyle{plainnat}
\bibliography{refs}

\end{document}
