% =============================
% Knowledge Compilation Map - Query Support Claims
% Auto-generated from database.json
% Generated: 2026-02-14T00:49:22.894Z
%
% EDITING INSTRUCTIONS:
% - Claims (\begin{claim}...\end{claim}) are auto-generated. Do NOT edit.
% - Descriptions (\begin{claimdescription}...\end{claimdescription}) are EDITABLE.
% - Derived entries are omitted; they will be regenerated by propagation.
% - To sync back to JSON, run: npx tsx scripts/latex-bijection.ts --to-json
% =============================
\documentclass[11pt]{article}

% -------- Packages --------
\usepackage[margin=1in]{geometry}
\usepackage{amsmath, amssymb, amsthm}
\usepackage{mathtools}
\usepackage{enumitem}
\usepackage{hyperref}
\usepackage{cleveref}
\usepackage{xcolor}
\usepackage{natbib}

% -------- Hyperref setup --------
\hypersetup{
  colorlinks=true,
  linkcolor=blue,
  citecolor=blue,
  urlcolor=blue
}

% -------- Theorem styles --------
\theoremstyle{plain}
\newtheorem{claim}{Claim}[section]

\theoremstyle{definition}
\newtheorem{definition}[claim]{Definition}

% -------- Description environment --------
\newenvironment{claimdescription}{%
  \par\noindent\ignorespaces
}{\par}

% -------- Handy macros --------
\newcommand{\R}{\mathbb{R}}
\newcommand{\N}{\mathbb{N}}
\newcommand{\eps}{\varepsilon}

% -------- Title info --------
\title{Knowledge Compilation Map: Query Support}
\date{\today}

\begin{document}
\maketitle

\tableofcontents
\newpage

% =============================
\section{A. Darwiche and P. Marquis, "A Knowledge Compilation Map," Journal of Artificial...}
% Reference ID: Darwiche_2002
% =============================
% lang=lang_89649e36, op=CO
\begin{claim}
$CNF$ supports Consistency not in polynomial time (quasi-polynomial unknown) \citet{Darwiche_2002}
\end{claim}
\begin{claimdescription}
(Description needed)
\end{claimdescription}

% lang=lang_4c204bf3, op=VA
\begin{claim}
$DNF$ supports Validity not in polynomial time (quasi-polynomial unknown) \citet{Darwiche_2002}
\end{claim}
\begin{claimdescription}
(Description needed)
\end{claimdescription}

% lang=lang_e02902d0, op=CO
\begin{claim}
$MODS$ supports Consistency in polynomial time \citet{Darwiche_2002}
\end{claim}
\begin{claimdescription}
A sentence $\Sigma$ is consistent iff it has at least one model. The number of models of $\Sigma$ is given by the number of edges outgoing from the or-node in any MODS representation of $\Sigma$. Accordingly, CO can be achieved in polynomial time by checking if this count is nonzero (Table 18, Darwiche 2002).
\end{claimdescription}

% lang=lang_e02902d0, op=CT
\begin{claim}
$MODS$ supports Model Counting in polynomial time \citet{Darwiche_2002}
\end{claim}
\begin{claimdescription}
The number of models of $\Sigma$ is given by the number of edges outgoing from the or-node in any MODS representation of $\Sigma$. Accordingly, CT (model counting) can be achieved in polynomial time by counting these edges (Table 18, Darwiche 2002).
\end{claimdescription}

% lang=lang_e02902d0, op=VA
\begin{claim}
$MODS$ supports Validity in polynomial time \citet{Darwiche_2002}
\end{claim}
\begin{claimdescription}
A sentence $\Sigma$ is valid iff it has $2^n$ models, where $n = |\text{Vars}(\Sigma)|$. The number of models of $\Sigma$ is given by the number of edges outgoing from the or-node in any MODS representation of $\Sigma$. Accordingly, VA can be achieved in polynomial time by checking if this count equals $2^n$ (Table 18, Darwiche 2002).
\end{claimdescription}

% lang=lang_b9d72a7c, op=CO
\begin{claim}
$OBDD$ supports Consistency in polynomial time \citet{Bryant_1986,Darwiche_2002}
\end{claim}
\begin{claimdescription}
Any query concerning OBDD is equivalent to the corresponding query concerning OBDD$_<$ when only one DAG is involved. Since OBDD$_<$ satisfies CO, VA and CT, so does OBDD \citet{Darwiche_2002}.
\end{claimdescription}

% lang=lang_b9d72a7c, op=CT
\begin{claim}
$OBDD$ supports Model Counting in polynomial time \citet{Bryant_1986,Darwiche_2002}
\end{claim}
\begin{claimdescription}
Any query concerning OBDD is equivalent to the corresponding query concerning OBDD$_<$ when only one DAG is involved. Since OBDD$_<$ satisfies CO, VA and CT, so does OBDD \citet{Darwiche_2002}.
\end{claimdescription}

% lang=lang_b9d72a7c, op=VA
\begin{claim}
$OBDD$ supports Validity in polynomial time \citet{Bryant_1986,Darwiche_2002}
\end{claim}
\begin{claimdescription}
Any query concerning OBDD is equivalent to the corresponding query concerning OBDD$_<$ when only one DAG is involved. Since OBDD$_<$ satisfies CO, VA and CT, so does OBDD \citet{Darwiche_2002}.
\end{claimdescription}

% lang=lang_d69995dd, op=CE
\begin{claim}
$OBDD_<$ supports Clausal Entailment in polynomial time \citet{Darwiche_2002}
\end{claim}
\begin{claimdescription}
(Description needed)
\end{claimdescription}

% lang=lang_d69995dd, op=IM
\begin{claim}
$OBDD_<$ supports Implicant in polynomial time \citet{Darwiche_2002}
\end{claim}
\begin{claimdescription}
(Description needed)
\end{claimdescription}

% lang=lang_d69995dd, op=SE
\begin{claim}
$OBDD_<$ supports Sentential Entailment in polynomial time \citet{Darwiche_2002}
\end{claim}
\begin{claimdescription}
(Description needed)
\end{claimdescription}


% =============================
\section{R. E. Bryant, "Graph-based algorithms for boolean function manipulation," Comput...}
% Reference ID: Bryant_1986
% =============================
% lang=lang_d69995dd, op=CO
\begin{claim}
$OBDD_<$ supports Consistency in polynomial time \citet{Bryant_1986}
\end{claim}
\begin{claimdescription}
Consistency is trivial for OBDDs: check if the diagram root is not the terminal 0 node \citet{Bryant_1986}.
\end{claimdescription}

% lang=lang_d69995dd, op=CT
\begin{claim}
$OBDD_<$ supports Model Counting in polynomial time \citet{Bryant_1986}
\end{claim}
\begin{claimdescription}
Model counting on OBDDs can be done in polynomial time by a single bottom-up pass \citet{Bryant_1986, Bryant_1992}.
\end{claimdescription}

% lang=lang_d69995dd, op=EQ
\begin{claim}
$OBDD_<$ supports Equivalence in polynomial time \citet{Bryant_1986,Bryant_1992,Meinel_Theobald_1998}
\end{claim}
\begin{claimdescription}
OBDDs with a fixed variable ordering are canonical: two functions are equivalent iff their reduced OBDDs are identical. This makes EQ trivial (structural isomorphism check) \citet{Bryant_1986, Bryant_1992, Meinel_Theobald_1998} (Theorem 8.11).
\end{claimdescription}

% lang=lang_d69995dd, op=ME
\begin{claim}
$OBDD_<$ supports Model Enumeration in polynomial time \citet{Bryant_1986}
\end{claim}
\begin{claimdescription}
(Description needed)
\end{claimdescription}

% lang=lang_d69995dd, op=VA
\begin{claim}
$OBDD_<$ supports Validity in polynomial time \citet{Bryant_1986}
\end{claim}
\begin{claimdescription}
Validity is trivial for OBDDs: check if the diagram root equals the terminal 1 node \citet{Bryant_1986}.
\end{claimdescription}


% =============================
\section{J. Gergov and C. Meinel, "Efficient Boolean manipulation with OBDD's can be exte...}
% Reference ID: Gergov_1994
% =============================
% lang=lang_684b1ca7, op=CO
\begin{claim}
$FBDD$ supports Consistency in polynomial time \citet{Gergov_1994}
\end{claim}
\begin{claimdescription}
It is well-known that FBDD satisfies CO \citet{Gergov_1994, Darwiche_2002}. Consistency is polynomial: check if the BDD root is not the terminal 0 node.
\end{claimdescription}

% lang=lang_684b1ca7, op=CT
\begin{claim}
$FBDD$ supports Model Counting in polynomial time \citet{Gergov_1994}
\end{claim}
\begin{claimdescription}
It is well-known that FBDD satisfies CT \citet{Gergov_1994, Darwiche_2002}. Model counting on FBDDs can be done in polynomial time by a single bottom-up pass counting satisfying assignments at each node.
\end{claimdescription}

% lang=lang_684b1ca7, op=VA
\begin{claim}
$FBDD$ supports Validity in polynomial time \citet{Gergov_1994}
\end{claim}
\begin{claimdescription}
It is well-known that FBDD satisfies VA \citet{Gergov_1994, Darwiche_2002}. Validity is polynomial: check if the BDD root equals the terminal 1 node.
\end{claimdescription}


% =============================
\section{C. Meinel and T. Theobald, "Algorithms and Data Structures in VLSI Design: OBDD ...}
% Reference ID: Meinel_Theobald_1998
% =============================
% lang=lang_b9d72a7c, op=SE
\begin{claim}
$OBDD$ supports Sentential Entailment not in polynomial time (quasi-polynomial unknown) \citet{Meinel_Theobald_1998}
\end{claim}
\begin{claimdescription}
Checking sentential entailment for OBDD formulas is coNP-complete. Since OBDD satisfies $\neg$C and since $\alpha \land \beta$ is consistent iff $\alpha \not\models \neg\beta$, checking SE for OBDD is coNP-complete (Lemma 8.14, Meinel \& Theobald 1998). This is because checking consistency of two OBDD$_<$ formulas with different variable orderings is NP-complete.
\end{claimdescription}


% =============================
\section{Adnan Darwiche, "Decomposable Negation Normal Form," Journal of the ACM, vol. 48...}
% Reference ID: Darwiche_2001a
% =============================
% lang=lang_3bebcab7, op=CE
\begin{claim}
$DNNF$ supports Clausal Entailment in polynomial time \citet{Darwiche_2001a}
\end{claim}
\begin{claimdescription}
DNNF supports clause entailment (CE) in polynomial time (Darwiche, 2001a).
\end{claimdescription}

% =============================
% Bibliography
% =============================
\bibliographystyle{plainnat}
\bibliography{refs}

\end{document}
