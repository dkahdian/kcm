% =============================
% Knowledge Compilation Map - Separating Functions
% Auto-generated from database.json
% Generated: 2026-02-18T00:54:53.270Z
%
% EDITING INSTRUCTIONS:
% - Short names in brackets are auto-generated identifiers. Do NOT edit.
% - Names (\textbf{...}) are EDITABLE (may contain LaTeX math).
% - Description content (after the name line) is EDITABLE.
% - To sync back to JSON, run: npx tsx scripts/latex-bijection.ts --to-json
% =============================
\documentclass[11pt]{article}

% -------- Packages --------
\usepackage[margin=1in]{geometry}
\usepackage{amsmath, amssymb, amsthm}
\usepackage{mathtools}
\usepackage{enumitem}
\usepackage{hyperref}
\usepackage{cleveref}
\usepackage{xcolor}
\usepackage{natbib}

% -------- Hyperref setup --------
\hypersetup{
  colorlinks=true,
  linkcolor=blue,
  citecolor=blue,
  urlcolor=blue
}

% -------- Theorem styles --------
\theoremstyle{definition}
\newtheorem{definition}{Definition}

% -------- Handy macros --------
\newcommand{\R}{\mathbb{R}}
\newcommand{\N}{\mathbb{N}}
\newcommand{\eps}{\varepsilon}

% -------- Title info --------
\title{Knowledge Compilation Map: Separating Functions}
\date{\today}

\begin{document}
\maketitle

\begin{definition}[$OR_n$]\label{sf:-or-n-}
\textbf{$OR_n = \bigvee_{i=1}^{n} x_i$} \\
Separates languages that compactly represent disjunctions from those that cannot. \citet{Darwiche_2002}
\end{definition}

\begin{definition}[Clique]\label{sf:clique}
\textbf{Clique Detection Instances} \\
Tests whether a directed graph on $n$ vertices (encoded as $I \in \{0,1\}^{n^2}$) contains a bidirectional clique of size $k$: $\Sigma_{n,k}(I) = 1 \iff I$ contains such a clique. Has $O(n^3)$ prime implicants but exponential $FBDD$ size for suitable $k$. \citet{Darwiche_2002,Wegener_1987}
\end{definition}

\begin{definition}[HWB]\label{sf:hwb}
\textbf{Hidden Weighted Bit} \\
Let $w = \sum_{i=1}^n x_i$ be the Hamming weight of the input. Then $HWB_n(x_1, \ldots, x_n) = x_w$ (the $w$-th input bit). Has $OBDD$ size $2^{\Omega(n)}$ but polynomial $SDD$ and $FBDD$ size. \citet{Bryant_1986}
\end{definition}

\begin{definition}[JS]\label{sf:js}
\textbf{Jukna-Schnitger function} \\
Tests triangle-freeness on graphs with respect to a fixed set of forbidden triangles $A \subset T_n$. \citet{Jukna_2002} show that for every $n$, there exists $A_n$ of size $O(n^2)$ such that any balanced rectangle cover of $JS_n^{A_n}$ has size $2^{\Omega(n^2)}$. Has $PI$ size $O(n^2)$ but $DNNF$ size $2^{\Omega(n^2)}$. \citet{Jukna_2002}
\end{definition}

\begin{definition}[Pairwise CNF]\label{sf:pairwise-cnf}
\textbf{$\Sigma_n = \bigwedge_{i=0}^{n-1} \left( x_{2i} \vee x_{2i+1} \right)$} \\
Requires that in each pair $(x_{2i}, x_{2i+1})$, at least one is true. Dual to the Pairwise DNF. \citet{Darwiche_2002}
\end{definition}

\begin{definition}[Pairwise DNF]\label{sf:pairwise-dnf}
\textbf{$\Pi_n = \bigvee_{i=0}^{n-1} \left( x_{2i} \wedge x_{2i+1} \right)$} \\
Requires that at least one pair $(x_{2i}, x_{2i+1})$ has both variables true. Dual to the Pairwise CNF. \citet{Darwiche_2002}
\end{definition}

\begin{definition}[Pairwise equivalence]\label{sf:pairwise-equivalence}
\textbf{$\Sigma_n = \bigwedge_{i=1}^n (x_i \leftrightarrow y_i)$} \\
Requires that each pair $(x_i, y_i)$ has equal truth values. An $OBDD_<$ with interleaved order $x_1 < y_1 < \cdots < x_n < y_n$ is polynomial, but a separated order $x_1 < \cdots < x_n < y_1 < \cdots < y_n$ forces exponential size. \citet{Darwiche_2002}
\end{definition}

\begin{definition}[Parity]\label{sf:parity}
\textbf{Parity function} \\
$O_n = \bigoplus_{i=0}^{n-1} x_i$ has linear size $OBBD_<$ representations but exponential size CNF and DNF representations. \citet{Bryant_1986}
\end{definition}

\begin{definition}[Sauerhoff]\label{sf:sauerhoff}
\textbf{Sauerhoff function} \\
Let $g_n: \{0,1\}^{2n} \to \{0,1\}$ be the function evaluating to 1 if and only if the sum of its inputs is divisible by 3. The Sauerhoff function $S_n: \{0,1\}^{n^2} \to \{0,1\}$ is defined on the $n \times n$ matrix $X = (x_{i,j})$ as $S_n(X) = R_n(X) \vee C_n(X)$, where $R_n, C_n : \{0,1\}^{n^2} \to \{0,1\}$ are defined by $R_n(X) = \bigplus_{i=1}^n g_n(x_{i1},x_{i2},...,x_{in}$ and $C_n(X) = R_n(X^T)$. \citet{Sauerhoff_2003,Bova_2016}
\end{definition}

% =============================
% Bibliography
% =============================
\bibliographystyle{plainnat}
\bibliography{refs}

\end{document}
